\documentclass[a4paper,11pt]{article}

\usepackage[utf8x]{inputenc}
\usepackage[T1]{fontenc}
\usepackage[francais]{babel}
\usepackage{amsmath,amssymb}
\usepackage{fullpage}
\usepackage{xspace}
\usepackage{verbatim}
\usepackage{graphicx}
\usepackage{listings}
\usepackage[usenames,dvipsnames]{color}
\usepackage{url}

\title{Rapport de projet d'algorithmique}
\author{Vivien Matter \and Ernest Foussard}
\date{Avril 2018}

% ===============
\begin{document}
% ===============
\maketitle

\tableofcontents

\newpage

\section{Introduction}

L'objectif du projet était de modifier des graphes de manière à ce qu'il soit
possible d'y trouver un cycle eulerien, puis de le calculer. Pour cela, on a été amenés
à implémenter plusieurs algorithmes et à s'interroger sur la complexité de ces derniers.

\section{Revue des algorithmes}

L'algorithme général se décompose en plusieurs sous-algorithmes: il faut d'abord reconnecter
les composantes connexes du graphe, puis ajouter des arêtes de telle sorte à ce que tous les
sommets soient de degré pairs, et enfin chercher un cycle eulerien.

\subsection{Reconnexion des composantes connexes}

\subsubsection{Algorithme quadratique}

L'algorithme naïf consiste à rechercher toutes les arêtes possibles avec l'ensemble de points
du graphe, de les trier par longueur, puis d'ajouter les segments qui relient deux composantes
connexes distinctes entre elles jusqu'à ce qu'il n'y ait plus qu'une seule composante connexe.
Soit $n$ le nombre de segments du graphe, cet algorithme admet une complexité temporelle $O(n²log(n))$.
On a implémenté une optimisation qui consiste à ne considerer que les arêtes qui relient deux composantes
initialement distinctes, ce qui se réalise facilement avec la structure d'union-find. La complexité dans
le pire des cas (cas où l'ensemble des arêtes est initialement vide) est toujours $O(n²log(n))$, mais
dans les situations pas trop tordues, le gain en temps est considerable.

\subsubsection{Algorithme avec table de hashage}

L'algorithme avec table de hashage consiste à construire un mappage spatial pour repérer les points par rapport
aux autres, et ensuite ajouter de manière gloutonne de manière quasiment ordonnée (on ne trouve pas nécessairement
les plus petits segments, mais les segments du même ordre de grandeur que le plus petit). Le gain en complexité temporelle
est considerable, a priori, et s'écrit $O(nlog(\frac{1}^{\alpha}))$ où $\alpha$ est la plus petite distance entre deux points du graphe.
Néanmoins, la complexité spatiale s'écrit $O(4^{\alpha})$ ce qui est extremement problématique dès qu'on a deux points très proches,
car le coût temporel pour instancier de tels objets devient alors très important.
Notre implémentation de cet algorithme ne s'execute pas en temps raisonnable pour la majorité des figures (c'est certainement dû
également à un manque d'optimisation ou une mécomprehension de l'algorithme).
\subsection{Construction du graphe de degrés pairs}

\subsubsection{Algorithme quadratique}

L'algorithme naïf consiste à parcourir l'ensemble des points une première fois pour compter les points de degré impairs,
puis on parcourt l'ensemble des couples de points du graphe par ordre croissant jusqu'à ce qu'il n'y ait plus de sommets de
degré impair, la complexité temporelle de cet algorithme est un $O(n²log(n))$. On a optimisé cet algorithme en ne triant et en n'
iterant que sur les sommets de degré impair. La complexité dans le pire des cas reste un $O(n²log(n))$ mais elle est général bien
moins importante (voir la partie perfomances).

\subsubsection{Algorithme avec table de hashage}

pas implémenté

\subsection{Détection d'un cycle eulerien}

Pour le calcul d'un cycle eulerien, on a implémenté l'agorithme d'Euler, qui admet une complexité temporelle $O(n)$. Cet algorithme consiste à passer sur des cycles quelconques du graphe en supprimant les segments
utilisés, et pour chaque point de ces cycles refaire cette manipulation. Le fait que le graphe soit à la fois connexe et que toutes
les arêtes soient de degré pairs nous assure la convergence et la correction du code. Comme on ne parcourt qu'une seule fois
chaque arrête du graphe, et que les opérations qu'on effectue lors du parcourt ne sont que des suppressions et des ajouts
de point dans une liste, qui sont instantanées, le passage de l'algorithme est bien en $O(n)$.

\section{Tests de performance}

hihihi.jpeg

\section{Conclusion}

pas fait

\end{document}
